\documentclass[11pt,a4paper,sans]{moderncv}

\moderncvstyle{classic}
\moderncvcolor{blue}

\nopagenumbers{}

\usepackage[utf8x]{inputenc} 
\usepackage[francais]{babel}

\usepackage[T1]{fontenc}

\usepackage[scale=0.9]{geometry}
\setlength{\hintscolumnwidth}{3.6cm}

%----------------------------------------------------------------------------------------

\firstname{Guillaume}
\familyname{Turchini}
\title{Développeur en systèmes et applications réparties}

\address{51B rue de la Haye}{78130 Les Mureaux}{}
%\mobile{06 16 17 37 90}
\email{guillaume.turchini@gmail.com}
\extrainfo{26 ans - Permis B}

%----------------------------------------------------------------------------------------

\begin{document}

\makecvtitle % Print the CV title
\vspace*{-0.5cm}

%----------------------------------------------------------------------------------------

\section{Expérience professionnelle}
\vspace{1em}

\cventry{Sept.~2016~--~Déc.~2018}
        {Ingénieur de développement Java Big Data}
        {Scaled Risk}{Paris. Langage : Java}{}
        {
            Amélioration et création de modules au sein de la Data Management Platform.
            \begin{itemize}
                \item Amélioration de l'ETL.
                \item Transformations et agrégations distribuées avec mise-à-jour via un bus temps-réel.
                \item Indexation bitemporelle de données via un coprocesseur HBase.
                \item Installation et administration de clusters Hadoop (HDP et MapR).
                \item Amélioration de la couverture des tests unitaires et d'intégration.
            \end{itemize}
        }

\vspace{1em}

\cventry{Oct. 2015 -- Juil. 2016}
        {Thèse}
        {LIP6 -- UPMC}{Paris}{}
        {
            \textbf{Sujet :} Plateforme passant à l'échelle pour les jeux massivement multijoueurs.\endgraf
            \textbf{Encadrant :} \textsc{Sébastien~Monnet}\hfill
            \textbf{Financement :} Bourse ministérielle\hfill~\endgraf
            \begin{itemize}
                \item\'Etude et élaboration d'algorithmes de répartition de charge et de stratégies de provisionnement dynamique dans des architectures de type Cloud.
                \item\'Etude de systèmes de type hybride Cloud et Pair-à-Pair.
                \item\'Elaboration d'un simulateur permettant d'estimer le coût d'une architecture Cloud.
            \end{itemize}
        }

\vspace{1em}

\cventry{Mars -- Août 2015}
        {Stage de fin d'étude}
        {LIP6}{Paris. Langage : Java}{}
        {
            Provisionnement dynamique de serveurs de jeu massivement multijoueurs.
            \begin{itemize}
                \item Création d'un modèle de mobilité de joueurs nécessaire à la simulation de serveur.
                \item\'Elaboration d'un simulateur pour des architectures de jeu massivement multijoueurs.
            \end{itemize}
        }

\vspace{1em}

\cventry{\'Eté 2014}
        {Stage de recherche dans le noyau Linux}
        {LIP6}{Paris. Langage : C}{}
        {
            \'Etude d'une anomalie de performance d'écriture sur disque rencontrée dans un sujet de thèse.
            \begin{itemize}
                \item\'Etude de la partie mm (memory management) du noyau Linux.
                \item\'Elaboration de benchmarks pour isoler le problème.
                \item\'Etude de traces du système d'écriture sur disque.
            \end{itemize}
        }

\vspace{1em}

\cventry{\'Eté 2013}
        {Stage de développement}
        {COHERIS}{Suresnes. Langages : Java, JavaScript}{}
        {
            \begin{itemize}
                \item Création d'un composant de prise de rendez-vous au sein de Coheris CRM.
                \item\'Etude de la portabilité de l'application Coheris CRM sous tablette windows 8.
                \item Développement d'un prototype d'application compagnon CRM sous Android.
            \end{itemize}
        }

\vspace{0.5em}

%----------------------------------------------------------------------------------------

\section{Formation}
\vspace{1em}

\cventry{2014--2015}
        {Master Systèmes et Applications Réparties - 2\up{ème} année}
        {UPMC}{Paris}{}
        {Mention Très Bien -- Major -- Co-habilité avec : \textsc{Telecom Paristech}}
\cventry{2013--2014}
        {Master Systèmes et Applications Réparties - 1\up{ère} année}
        {UPMC}{Paris}{}
        {Mention Très Bien}
\cventry{2012--2013}
        {Licence Informatique}
        {Université René Descartes}{Paris}{}
        {Mention Bien}
\cventry{2010--2012}
        {Classes préparatoires aux grandes écoles - MPSI, PSI\up{*}}
        {Lycée Condorcet}{Paris}{}
        {Option Sciences de l'Ingénieur}
\cventry{2009--2010}
        {Baccalauréat Scientifique}
        {Lycée François Villon}{Les Mureaux}{}
        {Mention Bien}

\vspace{2em}

%----------------------------------------------------------------------------------------

\section{Enseignements}
\vspace{1em}

\cvitem{Janv. -- Mai 2016}
       {Cours et Travaux pratiques sur python et théorie des graphes -- L1 -- 20 heures}
\cvitem{Sept. -- Déc. 2015}
       {Travaux pratiques de PR (Programmation Répartie / POSIX) -- M1 -- 44 heures}
\cvitem{2010 -- 2015}
       {Professeur particulier en matières scientifiques niveau lycée}

\vspace{0.5em}

%----------------------------------------------------------------------------------------

\section{Compétences}
\vspace{1em}

\cvitem{Langages favoris}
       {
            Java,
            C,
            Rust,
            Scala
       }
\cvitem{Autres Langages}
       {
            Bash,
            C++,
            Python,
            R,
            JavaScript,
            LaTeX
       }
\cvitem{Systèmes}
       {
            Linux (Arch Linux, Debian, CentOS),
            Windows,
            Android
       }
\cvitem{Outils utilisés}
       {
            IntelliJ,
            Maven,
            Git,
            Docker,
            Yourkit,
            Visual Studio Code
       }
\cvitem{Modélisation}
       {
            UML,
            Design Patterns
       }
\cvitem{Base de données}
       {
            HBase,
            (Postgre)SQL,
            Hibernate
       }
\cvitem{Technologies}
       {
            Hadoop (HDP et MapR),
            Coprocesseur HBase,
            Lucene,
            Tomcat,
            JUnit / TestNG,
            Spring
       }
\cvitem{Algorithmique}
       {
            Décision répartie,
            P2P,
            Détection de fautes,
            MapReduce (Spark / Yarn)
       }
\cvitem{Intégration continue}
       {
            TeamCity,
            AppVeyor,
            SonarQube,
            Coveralls
       }

\vspace{0.5em}

%----------------------------------------------------------------------------------------

\section{Langues}
\vspace{1em}

\cvitem{Français}{Langue maternelle}
\cvitem{Anglais}{Lu, écrit, parlé - Général et technique}

\vspace{0.5em}

%----------------------------------------------------------------------------------------

\section{Divers}
\vspace{0.3em}

\subsection{Publications}
\vspace{1em}

\cvitem{Fast Abstract}
       {
            Scalability and availability for massively multiplayer online games.\endgraf
            \textsc{Guillaume~Turchini, Sébastien~Monnet, Olivier~Marin}.\endgraf
            In \textit{European Dependable Computing Conference 2015} (EDCC'15).
       }
       
\vspace{0.5em}

\subsection{Activités collaboratives}
\vspace{1em}

\cvitem{Sept. 2015}
       {Participation à l'organisation de la conférence EDCC 2015}
\cvitem{2014}
       {\textit{Membre de la team} au sein de l'association étudiante AEIP6 (UPMC)}
\cvitem{2012-2017}
       {Installation bénévole de distributions Linux (Install party)}
\cvitem{2012-2013}
       {\textit{Membre actif} au sein de l'association étudiante MIBDE (Université Paris Descartes)}

\end{document}
